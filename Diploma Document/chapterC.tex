\chapter{Design}\label{ch:design}
\section{Tools}
	\subsection{Figma}
	Figma is the main tool of the UX/UI designer. On this service we have developed the design of our website. In figma we can:
	\begin{itemize}
		\item Develop the interface and prototype of the site
		\item Save files to the cloud
		\item Integrate with different design applications
		\item Several people can enter and edit at the same time
	\end{itemize}
	\subsection{Adobe Illustrator}
	This application was needed to create the logo for our game. This application makes it possible to:
	\begin{itemize}
		\item Easy to create vector design
		\item Save it in any convenient format
		\item Save as best quality
	\end{itemize}
	
\section{Graphic Design}
This is the main part of the design. At this stage, we already decide on colours, fonts and patterns. Graphic design helps to build the right communication with users. The visual concept begins to be developed at this stage and the most important part of graphic design is the logo.
	\subsection{Logo}
	As you can see, the logo consists of three figures. Each figure represents a player, and by this we want to make it clear that this is a multiplayer game. The logic behind the figures is that they are a connection between two points and represent the player's movement from one point to the other, which is the main function of our game.

\section{UX/UI design}
In design, it is very important that the structure is created according to logic and all the rules of design. No matter how beautiful the site would be, if the structure is created incorrectly, the site will be non-working.

	\subsection{Main page}
		\subsubsection{Header}
		\begin{itemize}
			\item Our logo is located on the upper left side.
			\item "About us", "Contact", "Log in" will be located on the upper right side
		\end{itemize}
		\subsubsection{Onboarding}
		\begin{itemize}
			\item Also on the main page there is a brief information about the game, this is done so that when a person first visits the site, he understands where he got to.
		\end{itemize}
		\subsubsection{Starting game}
		\begin{itemize}
			\item In the middle there will be 2 main buttons "Start game", "Join the game".
			\item "Start game". The emphasis is on the "Start game" button by coloring the button with a bright color. The emphasis was on this button in order to encourage more users to register in the game, since when you click on the "Start game" button, if a person is not registered, a registration window will pop up for him. And if a person is registered, then when you click on this button, he can immediately start playin.
			\item Join the game". This button is designed to allow players to join the creator's game. Clicking on this button will pop up a window with an input where players could enter the id of the game they want to join.
		\end{itemize}
		\subsubsection{Footer}
		\begin{itemize}
			\item Footer will contain links to social networks in the form of icons
		\end{itemize}